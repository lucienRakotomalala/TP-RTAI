\chapter{TP 1 : Iniiation a un OS temps Réel basé sur Linux}

\section{Mesures sous linux}
Void out envoie un signal carré. Frequence = définie dans main
		Amplitude de 0 1à 1

dans la main est init une structure de temps, est ensuite ouvert la carte entrée sortie.la carte est paramétrée en sortie sur les ports 0 et 1. 
ensuite, l'algorithme attend 
initialisation d'une horloge qui va atenddre un temps correspondant a la demi-période  du signal carré généré


Pou mesurer les modifications de période, nous avons crée deux variables $timespec$ : une qui mesure le temps précédent le sleep, une qui mesure a la fin de l'instance $while(1)$. La mesure de la $\delta$ est : $\delta = t_{debut} - t_{fin}$. 

Mise en place d'un $gnuplot$ pour afficher les 5000 dernières périodes. 

Observation : \begin{itemize}
\item pour aucune charges de linux, les périodes restent à $50\mu s$. 
\item pour un simple 
\end{itemize}

\section{Mesures sous RTAI}

Utilisation de la ligne : 
\begin{lstlising}
cjz
\end{lstlisting}

