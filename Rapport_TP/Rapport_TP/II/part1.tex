\section{Position du problème}
Nous avons décider de séparer notre logiciel en 2 tâches temps réel : \begin{itemize}
\item \emph{void generateur} : fonction de haute priorité qui s'occupe de générer les 2 signaux sinusoïdaux. 
\item \emph{void lecture} : fonction qui va lire les valeurs en entrées (potentiomètre et numérique) pour recalculer les signaux.
\end{itemize}
Au vue de ces deux tâches, il est impératif de décider maintenant de la différence de priorité entre ces deux taches principale. Il nous semble approprier, à ce stade de l'implémentation, de donner plus de priorité à la fonction \emph{void generateur} pour permettre un rafraichissement de la valeur envoyé sur le convertisseur numérique analogique suffisante. La fonction \emph{void lecture} sera alors appelée moins régulièrement que la génération des signaux.
\subsection{Génération du signal périodique}
Les calculs dans un programme temps réel ne sont généralement pas souhaitables : il demande trop de temps, selon leur complexité, au processeur. Selon cette affirmation, pour générer les signaux sinusoïdaux, nous ne devons faire aucun calcul de $\sin$ dans le programme principal. Nous devons générer d'une autre manière un série de 50 valeurs comprises entre [A VERIFIER] qui représente notre signal sinusoïdal qui n'a pas subit de modifications.

SA SERAIT COOL D'ajouter un graphique
\subsection{Conversion des valeurs analogique vers numérique}
Pour pouvoir sortir un signal analogique, le programme RTAI utilise un convertisseur numérique-analogique 16 bits avec la librairie \emph{comedio}. Il nous faut adapter les valeurs en V que nous souhaitons envoyer à ces valeurs numériques. Nous allons utiliser la loi suivante qui a été calculé en sachant que le signal peut avoir une amplitude max de [-10V; 10V]. Avec ces informations, nous trouvons : \begin{align*}
S_{data} = \frac{65535}{20}\times V + 32767
\end{align*}

\subsection{Envoie des signaux sur le CNA}
L'implémentation de la tâche de haute priorité nécessite de commencer par l'initialisation RTAI de celle-ci. Nous décidons de mettre une fréquence fixe pour commencer de 1ms 

\subsection{Observation des résultats}
