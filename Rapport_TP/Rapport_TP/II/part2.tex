\section{Implémentation de la modification des signaux}

\subsection{Division de la tâches \emph{void generateur}}	
Il est nécessaire de modifier le code précédent pour respecter les contraintes générales. Celles-ci nous demandent de donner la possibilité de modifier la fréquence des signaux qui dépend elle-même de la fréquence de l'unique tâche \emph{generateur}. Pour modifier la fréquence d'un seul signal de manière indépendante, il est donc nécessaire de séparer a tâche \emph{generateur} en deux taches temps réel \emph{genrateur1} et \emph{genrateur2}. 

Problèmes de INIT que j'ai pas compris
\subsection{Tâche de lecture des informations}
Maintenant que les signaux sont indépendants, nous pouvons commencer l'implémentation de la tâche mons récurrente : \emph{void lecture}. Cette fonction va utiliser plusieurs types de signaux : 
\begin{itemize}
\item[\textbf{a} :]valeur analogique pour modifier l'amplitude.
\item[\textbf{p} :]valeur analogique pour modifier la phase.
\item[\textbf{f} :]valeur analogique pour modifier la fréquence.
\item[\textbf{n} :]valeur binaire pour sélectionner le signal à modifier.
\item[\textbf{a} :]valeur binaire pour sélectionner la sensibilité.
\end{itemize}

\subsection{Application de ces informations aux signaux}


\subsection{Observation des résultats}